\documentclass[11pt]{beamer}
\usetheme{Warsaw}
\usepackage[utf8]{inputenc}
\usepackage[english]{babel}
\usepackage{amsmath}
\usepackage{amsfonts}
\usepackage{amssymb}
\usepackage{graphicx}
%\author{}
%\title{}
%\setbeamercovered{transparent} 
%\setbeamertemplate{navigation symbols}{} 
%\logo{} 
%\institute{} 
%\date{} 
%\subject{} 
\begin{document}

%\begin{frame}
%\titlepage
%\end{frame}

\begin{frame}
\tableofcontents
\end{frame}

\begin{frame}{Motivation}
\begin{itemize}[<+->]
\item Differential abundance microbiome studies are generally underpower \cite{kers2021power} %; @brussow2020problems] 
\item Significance results reported  in the literature generally do not account for Type 1 and Type 2 error 
\item Difficult to decide on average effect size and power to use when conducting power analysis
\item Possible remedy: gain an understanding into the relationships among effect sizes, power and abundance 
    \end{itemize}
\end{frame}

\begin{frame}{Research Questions}
The goal of this project is to investigate
\begin{itemize}[<+->]
\item  the relationships among power, effect sizes and abundances with the aim of understanding the mechanisms behind these relationships.
\item the number of individual taxa in a differential abundance
studies that power can be detected reliably, for a given effect and sample size.
\item the number of sample sizes required to detect a given power and a given effect size. 
    \end{itemize}

\end{frame}

\begin{frame}{Method} 
\begin{itemize}[<+->]
\item Collected and processed 10 datasets from projects that studied the microbome of children with autism spectral disorder
\href{https://www.ebi.ac.uk/ena/browser/view}{ENB}
 \href{https://www.ncbi.nlm.nih.gov/}{NCBI}
 \item  Data was processed using the Dada2 pipeline into Amplicon Sequence Variant (ASV) table
\end{itemize}
\end{frame}


\begin{frame}
\includegraphics[scale=0.5]{}

\end{frame}


























\begin{frame}[allowframebreaks]
        \frametitle{References}
        \bibliographystyle{amsalpha}
        \bibliography{packages.bib}
\end{frame}


\end{document}